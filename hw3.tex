%{
\documentclass[11pt, twoside]{article}   	% use "amsart" instead of "article" for AMSLaTeX format
\usepackage{geometry}                		% See geometry.pdf to learn the layout options. There are lots.
\geometry{letterpaper}                   		% ... or a4paper or a5paper or ... 
%\geometry{landscape}                		% Activate for for rotated page geometry
\usepackage[parfill]{parskip}    		% Activate to begin paragraphs with an empty line rather than an indent
\usepackage{graphicx}				% Use pdf, png, jpg, or eps§ with pdflatex; use eps in DVI mode
								% TeX will automatically convert eps --> pdf in pdflatex	
\usepackage{amssymb}
\usepackage{color}
\usepackage{matlab-prettifier}
\usepackage{verbatim}
\usepackage{fancyvrb}
\usepackage{multicol}
\usepackage{amsmath}
\usepackage{cite}
\usepackage{bm}
\usepackage{hyperref} 
\usepackage[section]{placeins}
\lstset{style=Matlab-editor,basicstyle=\ttfamily}

\sloppy
\definecolor{lightgray}{gray}{0.5}
\newenvironment{matlab}{\comment}{\endcomment}
\newenvironment{matlabv}{\lstlisting}{\endlstlisting}	


\title{Homework \# 3 \\ Gaussian Processes}
\author{Cody W. Eilar}
%\date{}							% Activate to display a given date or no date

\begin{document}

\maketitle


\begin{matlab}
%}
close all; 
clear all;

fid = fopen('output_dir/computer.tex', 'w'); 
fprintf(fid, computer); 
fclose(fid); 

fid = fopen('output_dir/matlabver.tex', 'w'); 
a = ver('matlab'); 
fprintf(fid, [a.Name ' version ' a.Version]); 
fclose(fid); 
%{
\end{matlab}

\section{Introduction} 
In the previous homework assignments we focused on processes with support vector machines
and kernels. Support vector machines are great because no \textit{a prior} assumptions
are made about the data. This then allows them to work in a myriad of cases when a
distribution about the data is not known. However, they can become computationally 
infeasible when cross validation is introduced. When dealing with large datasets, it can
take hours or days to properly compute the accuracy of a machine using cross validation 
techniques. This is where Gaussian processes are truly a useful tool. With Gaussian processes
there is no need to cross validate because the uncertainty is built into the model and it 
is therefore safe to assume with some certainty that the data in which you are trying to 
predict or fit will fall into that assumption.

In this paper we will explore how Linear Gaussian Processes work and what their advantages, 
as well as their disadvantages are for regressing and classifying stochastic processes. 
We will also look into extending Gaussian Processes to using \textit{Recursive Kernel Hilbert
Spaces}. We then look at how to optimize the hyper-parameters using exact inference. 
This then will set us up to demonstrate this theory in a few experiments that use
\textbf{MATLAB} to demonstrate a few key principles of Gaussian Processes. 
\section{Theory}
\subsection{Linear Gaussian Process}
\subsection{Gaussian process in Recursive Kernel Hilbert Spaces}
\subsection{Inference over the hyperparameters}

\section{Methodology and Experiments}
\subsection{Linear Gaussian Process}
\subsubsection{Experimental setup}
\subsubsection{Linear Gaussian Process for Prediction}
\subsubsection{Linear ARMA and AR(1) process noise}
\subsection{Nonlinear Gaussian Process}

\section{Discussion and Conclusion}

\bibliography{hw3.bib}
\bibliographystyle{ieeetr}

\end{document}  
%}

